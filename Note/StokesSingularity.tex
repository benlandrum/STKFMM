\documentclass[11pt]{article}
\usepackage[left=1in, right=1in, top=1in, bottom=1in]{geometry}
\usepackage[T1]{fontenc}
\usepackage{stix}
\usepackage{amsmath}
\usepackage{amsfonts}
\usepackage{graphicx}
\usepackage{bm}
\usepackage{natbib}
\usepackage[section]{placeins}

\author{Wen Yan}

\newcommand{\AVE}[1]{\ensuremath{\langle {#1} \rangle}}
\newcommand{\ABS}[1]{\ensuremath{\lvert {#1} \rvert}}
\newcommand{\dpone}[2]{\ensuremath{\displaystyle\frac{\partial {#1}}{\partial {#2}}}}
\newcommand{\dptwo}[2]{\ensuremath{\displaystyle\frac{\partial^2 {#1}}{\partial {#2}^2}}}
\newcommand{\dpn}[3]{\ensuremath{\displaystyle\frac{\partial^{#1} {#2}}{\partial {#3}^{#1}}}}

\newcommand{\bA}{\ensuremath{\bm{A}}}
\newcommand{\bB}{\ensuremath{\bm{B}}}
\newcommand{\bC}{\ensuremath{\bm{C}}}
\newcommand{\bD}{\ensuremath{\bm{D}}}
\newcommand{\bE}{\ensuremath{\bm{E}}}
\newcommand{\bF}{\ensuremath{\bm{F}}}
\newcommand{\bG}{\ensuremath{\bm{G}}}
\newcommand{\bH}{\ensuremath{\bm{H}}}
\newcommand{\bI}{\ensuremath{\bm{I}}}
\newcommand{\bJ}{\ensuremath{\bm{J}}}
\newcommand{\bK}{\ensuremath{\bm{K}}}
\newcommand{\bL}{\ensuremath{\bm{L}}}
\newcommand{\bM}{\ensuremath{\bm{M}}}
\newcommand{\bN}{\ensuremath{\bm{N}}}
\newcommand{\bO}{\ensuremath{\bm{O}}}
\newcommand{\bP}{\ensuremath{\bm{P}}}
\newcommand{\bQ}{\ensuremath{\bm{Q}}}
\newcommand{\bR}{\ensuremath{\bm{R}}}
\newcommand{\bS}{\ensuremath{\bm{S}}}
\newcommand{\bT}{\ensuremath{\bm{T}}}
\newcommand{\bU}{\ensuremath{\bm{U}}}
\newcommand{\bV}{\ensuremath{\bm{V}}}
\newcommand{\bW}{\ensuremath{\bm{W}}}
\newcommand{\bX}{\ensuremath{\bm{X}}}
\newcommand{\bY}{\ensuremath{\bm{Y}}}
\newcommand{\bZ}{\ensuremath{\bm{Z}}}


\newcommand{\ba}{\ensuremath{\bm{a}}}
\newcommand{\bb}{\ensuremath{\bm{b}}}
\newcommand{\bc}{\ensuremath{\bm{c}}}
\newcommand{\bd}{\ensuremath{\bm{d}}}
\newcommand{\be}{\ensuremath{\bm{e}}}
\newcommand{\bff}{\ensuremath{\bm{f}}}
\newcommand{\bg}{\ensuremath{\bm{g}}}
\newcommand{\bh}{\ensuremath{\bm{h}}}
\newcommand{\bi}{\ensuremath{\bm{i}}}
\newcommand{\bj}{\ensuremath{\bm{j}}}
\newcommand{\bk}{\ensuremath{\bm{k}}}
\newcommand{\bl}{\ensuremath{\bm{l}}}
\newcommand{\bmm}{\ensuremath{\bm{m}}}
\newcommand{\bn}{\ensuremath{\bm{n}}}
\newcommand{\bo}{\ensuremath{\bm{o}}}
\newcommand{\bp}{\ensuremath{\bm{p}}}
\newcommand{\bq}{\ensuremath{\bm{q}}}
\newcommand{\br}{\ensuremath{\bm{r}}}
\newcommand{\bs}{\ensuremath{\bm{s}}}
\newcommand{\bt}{\ensuremath{\bm{t}}}
\newcommand{\bu}{\ensuremath{\bm{u}}}
\newcommand{\bv}{\ensuremath{\bm{v}}}
\newcommand{\bw}{\ensuremath{\bm{w}}}
\newcommand{\bx}{\ensuremath{\bm{x}}}
\newcommand{\by}{\ensuremath{\bm{y}}}
\newcommand{\bz}{\ensuremath{\bm{z}}}

\newcommand{\bsigma}{\ensuremath{\bm{\sigma}}}

\begin{document}
\title{Singularity solutions in Stokes flow}
\maketitle
\section{Introduction}
The purpose of this document is to clear up the chaotic mess in literatures about Stokes flow singularities including the name, sign, index, notation.

This document is mostly based on the following materials:
\begin{itemize}
	\item 1.Blake, J. R. \& Chwang, A. T. Fundamental singularities of viscous flow. J Eng Math 8, 23-29 (1974).
	\item 2.Durlofsky, L., Brady, J. F. \& Bossis, G. Dynamic Simulation of Hydrodynamically Interacting Particles. Journal of Fluid Mechanics 180, 21-49 (1987).
	\item 3.Kim, S. \& Karrila, S. J. Microhydrodynamics: Principles and Selected Applications. (Courier Corporation, 2005).
	\item STFMM3DLib document
\end{itemize}

\textbf{Important Remark: }The force, torque, stresslet in this document are all applied on the fluid.

\section{Definitions}
The Stokes equation:
\begin{align}
	\nabla\cdot\bsigma = -\nabla p + \mu \nabla^2 \mu &= -\bF \delta(\bx_F) \\
	\nabla\cdot\bu &=0
\end{align}

Define the vector $\br$ pointing from source to target, and the $G_{ij}$ Oseen-Burgers tensors:
\begin{align}
	G_{ij} &= \frac{\delta_{ij}}{r} + \frac{r_ir_j}{r^3}\\
	G_{ij,k} &= \left[ \frac{\delta_{jk}r_i}{r^3} - \frac{3r_ir_jr_k}{r^5} \right] + \left[\frac{\delta_{ik}r_j}{r^3} -\frac{\delta_{ij}r_k}{r^3} \right] \\
	\nabla^2 G_{ij} &= 2\frac{\delta_{ij}}{r^3} - 6\frac{r_ir_j}{r^5} = -2\nabla\nabla \frac{1}{r}
\end{align}
Note that $G_{ij,k}$ contains a symmetric and an anti-symmetric part for index $j,k$. We need that in the derivation for doublet, stresslet, and rotlet. 

In the next section we consider a sphere with radius $a$, and consider the force exerted by this sphere on the fluid. Let $n$ denotes the norm vector pointing outward the sphere. We have:
\begin{align}
	F_i &= \int \sigma_{ij} n_j dS \\
	L_i &= \int \epsilon_{ijk} y_j \sigma_{kl} n_l dS \\
	D_{ij} &= \int \sigma_{ik} n_k y_j dS \\
	S_{ij} &= \frac{1}{2}\left(D_{ij}+D_{ji}\right) - \frac{1}{3}D_{kk}\delta_{ij}\\
	T_{ij} &= \frac{1}{2}\left(D_{ij}-D_{ji}\right)
\end{align}
$S_{ij}$ is symmetric, and $T_{ij}$ is anti-symmetric. Both of them are trace-free tensors. Further, $D_{ij},S_{ij},T_{ij}$ refers to force strength with index $i$ and direction with index $j$.

\section{Propagators for finite size sphere $a$.}
This part is taken from Durlofsky, 1987.
\begin{align}
	u_i &= \frac{1}{8\pi\mu} \left(1+\frac{1}{6}a^2\nabla^2 \right) G_{ij}F_j + \frac{1}{8\pi\mu} R_{ij} L_j + \frac{1}{8\pi\mu} \left(1+\frac{1}{10}a^2\nabla^2 \right) K_{ijk}S_{jk}
\end{align}
Here 
$$ K_{ijk} = \frac{1}{2}\left(G_{ij,k}+G_{ik,j}\right) = \frac{\delta_{jk}r_i}{r^3} - \frac{3r_ir_jr_k}{r^5} . $$
$$ R_{ij} =  \frac{\epsilon_{ijk}r_k}{r^3}.$$
By utilizing the trace-free property of $S_{jk}$ by definition, we have $S_{jk}\delta_{jk}=0$. So the flow disturbance can also be written as:
\begin{align}
	\left(1+\frac{1}{10}a^2\nabla^2 \right)\left(- \frac{3r_ir_jr_k}{r^5}\right)S_{jk}
\end{align}



The torque $L_j$ is related to the rotlet $T_{ij}$ as:
\begin{align}
	T_{ij}&=\frac{1}{2}\epsilon_{ijk}L_k\\
	L_i &= \epsilon_{ijk} T_{jk}  
\end{align}

\subsection{Faxen's Laws}
\begin{align}
	U_i - u_i^\infty &= \frac{F_i}{6\pi\mu a} + \left(1+\frac{1}{6}a^2\nabla^2\right)u_i' \\
	\Omega_i - \Omega_i^\infty &= \frac{L_i}{8\pi\mu a^3} + \frac{1}{2}\epsilon_{ijk}\nabla_j u_k' \\
	-E_{ij}^\infty &= \frac{S_{ij}}{\frac{20}{3}\pi\mu a^3} +\left(1+\frac{1}{10}a^2\nabla^2\right) e_{ij}'
\end{align}

\section{Limit of point singularities: $a\to 0$.}
\subsection{Velocity}
This is equivalent to setting $a=0$ in Eq.(11). But here we rederive it to clear up things. We start from the point-multipole expansion. The force term is straightforward and the doublet term is:
\begin{align}
	u_i &= \frac{1}{8\pi\mu} G_{ij,k} D_{jk} \\
	&= \frac{1}{8\pi\mu} \left(\left[ \frac{\delta_{jk}r_i}{r^3} - \frac{3r_ir_jr_k}{r^5} \right] + \left[\frac{\delta_{ik}r_j}{r^3} -\frac{\delta_{ij}r_k}{r^3} \right]\right)D_{jk}
\end{align}
The propagators for $S_{ij}$ and $T_{ij}$ can be constructed as follows:
\begin{align}
	\frac{1}{2} u_i &= \frac{1}{8\pi\mu} \frac{1}{2} D_{jk} G_{ij,k} \\
	\frac{1}{2} u_i &= \frac{1}{8\pi\mu} \frac{1}{2} D_{kj} G_{ik,j}
\end{align}
Add them together, utilizing that $G_{ij,k}$ contains a symmetric and an anti-symmetric part:
\begin{align}
	u_i &= \frac{1}{8\pi\mu} \frac{1}{2}\left(D_{jk}+D_{kj}\right)\left[ \frac{\delta_{jk}r_i}{r^3} - \frac{3r_ir_jr_k}{r^5} \right] + \frac{1}{8\pi\mu} \frac{1}{2}\left(D_{jk}-D_{kj}\right) \left[\frac{\delta_{ik}r_j}{r^3} -\frac{\delta_{ij}r_k}{r^3} \right]\\ 
	&= \frac{1}{8\pi\mu}S_{jk}\left[ \frac{\delta_{jk}r_i}{r^3} - \frac{3r_ir_jr_k}{r^5} \right] + \frac{1}{8\pi\mu}\left( \frac{2}{3} D_{ii}\delta_{jk} \right)\left[ \frac{\delta_{jk}r_i}{r^3} - \frac{3r_ir_jr_k}{r^5} \right] + \frac{1}{8\pi\mu}T_{jk} \left[\frac{\delta_{ik}r_j}{r^3} -\frac{\delta_{ij}r_k}{r^3} \right]
\end{align}

It is clear that the trace-dependent term is zero:
\begin{align}
   \delta_{jk} \left[ \frac{\delta_{jk}r_i}{r^3} - \frac{3r_ir_jr_k}{r^5} \right] = 3\frac{r_i}{r^3} - 3\frac{r_ir_jr_j}{r^5} = 0
\end{align}

$S_{jk}$ is trace-free so $S_{jk}\delta_{jk}=0$. Therefore:
\begin{align}
	u_i = \frac{1}{8\pi\mu}S_{jk}\left[ - \frac{3r_ir_jr_k}{r^5} \right]  + \frac{1}{8\pi\mu}T_{jk} \left[\frac{\delta_{ik}r_j}{r^3} -\frac{\delta_{ij}r_k}{r^3} \right]
\end{align}

We further utilize the relation between the torque $L_j$ and the rotlet $T_{jk}$, we have:
\begin{align}
	T_{jk} \left[\frac{\delta_{ik}r_j}{r^3} -\frac{\delta_{ij}r_k}{r^3} \right] &= \frac{1}{2}\epsilon_{jkl}L_l \left[\frac{\delta_{ik}r_j}{r^3} -\frac{\delta_{ij}r_k}{r^3} \right] = \frac{1}{2} L_l \left(\epsilon_{jil}r_j - \epsilon_{ikl}r_k\right) \frac{1}{r^3} \\
	&= L_l \frac{\epsilon_{ilj}r_j}{r^3}
\end{align}

We get back to the same equation by setting $a=0$ in Eq.(11).

\textbf{Remark: } The finite-size effect factor $a^2\nabla^2$ appears for stresslet $S_{jk}$ but not for torque $L_j$ in Eq.(11) is due to the fact that $\nabla^2 L_l \frac{\epsilon_{ilj}r_j}{r^3} =0 $.

\textbf{Remark 2:} The equation in Blake 1971 and STFMM3D for $D_{jk}$ seems to be off by a negative sign. 

\subsection{Pressure}
\begin{align}
p &= \frac{1}{4\pi} \frac{r_k}{r^3} F_k + \frac{1}{4\pi}\left(-3\frac{r_jr_k}{r^5} + \frac{\delta_{jk}}{r^3}\right)D_{jk} \\
	& = \frac{1}{4\pi} \frac{r_k}{r^3} F_k + \frac{1}{4\pi}\left(-3\frac{r_jr_k}{r^5}\right)S_{jk}\\
	\dpone{p}{x_i} &= \frac{1}{4\pi}\frac{r^2 F_i - 3r_ir_kF_k}{r^5} + \frac{1}{4\pi} \left(15\frac{r_ir_jr_k}{r^7}-3\frac{\delta_{ij}r_k+\delta_{ik}r_j+\delta_{jk}r_i}{r^5}\right) D_{jk} \\
	&= \frac{1}{4\pi}\frac{r^2F_i - 3r_ir_kF_k}{r^5} + \frac{1}{4\pi} \left(15\frac{r_ir_jr_k}{r^7}-3\frac{\delta_{ij}r_k+\delta_{ik}r_j}{r^5}\right) S_{jk}
\end{align}
Since $S_{jk}$ is symmetric and trace-free.
Torque/Rotlet does not contribute to pressure disturbance.

\subsection{Double layer potential}
The following two combinations satisfy the Stokes equation, for trace-free $S_{jk}$ and general $D_{jk}$, respectively:
\begin{align}
	p& =\frac{1}{4\pi}\left(-3\frac{r_jr_k}{r^5}\right)S_{jk}, \quad u_i = \frac{1}{8\pi\mu}\left[ - \frac{3r_ir_jr_k}{r^5} \right] S_{jk} \\
	p& =\frac{1}{4\pi}\left(-3\frac{r_jr_k}{r^5} + \frac{\delta_{jk}}{r^3}\right)D_{jk}  ,\quad u_i = \frac{1}{8\pi\mu} \left(\left[ \frac{\delta_{jk}r_i}{r^3} - \frac{3r_ir_jr_k}{r^5} \right] + \left[\frac{\delta_{ik}r_j}{r^3} -\frac{\delta_{ij}r_k}{r^3} \right]\right)D_{jk}
\end{align}

For $S_{jk}$, the Stokes equation generates:
\begin{align}
	-\dpone{p}{r_i} + \nabla^2 u_i = -\frac{3}{4}\frac{r_i}{r^5} S_{jj}
\end{align}
Because $S_{jj}$ is trace-free by definition, Stokes equation is satisfied. Note that the satisfaction of Stokes equation does not require the symmetry of $S_{jk}$.

For $D_{jk}$ the Stokes equation is always zero, since $D_{jk}$ is arbitrary by definition.

\textbf{Important: }In boundary integral theory the double layer kernel $\frac{1}{8\pi\mu}S_{jk}\left[ - \frac{3r_ir_jr_k}{r^5} \right]$ is used with an arbitrary source $v_kg_j$, where the trace-free condition is not necessarily satisfied. In this case, the pressure calculated by $\frac{1}{4\pi}\left(-3\frac{r_jr_k}{r^5}\right)v_kg_j$ is not correct.

For arbitrary $D_{jk}$ or arbitrary $D_{jk}=v_kg_j$, the following combination of pressure and velocity kernel still satisfies the Stokes equation:
\begin{align}\label{eq:doublelayerfmm}
		p& =\frac{1}{4\pi}\left(-3\frac{r_jr_k}{r^5} + \frac{\delta_{jk}}{r^3}\right)D_{jk}, \quad u_i = \frac{1}{8\pi\mu}\left[ - \frac{3r_ir_jr_k}{r^5} \right] D_{jk}
\end{align}
This is the correct formula which should be used in the double layer boundary integral equations (cf. STFMMLIB documentation).


\subsection{Derivatives of double layer potential}
Grad and Laplacian, where $\nabla^2 \bu = \nabla p$ due to Stokes equation:
\begin{align}
	\left[\frac{r_ir_jr_k}{r^5} \right]_{,l} &= \delta_{il}r_jr_k\frac{1}{r^5} + \delta_{jl}r_ir_k\frac{1}{r^5} + \delta_{kl}r_ir_j\frac{1}{r^5} + r_ir_jr_k\left(-5\frac{r_l}{r^7}\right) \\
	\left[\frac{r_ir_jr_k}{r^5} \right]_{,ll} &= 2\left[\delta_{ij}r_k\frac{1}{r^5}+\delta_{ik}r_j\frac{1}{r^5}+\delta_{jk}r_i\frac{1}{r^5}\right]-10\left[\frac{1}{r^7}r_ir_jr_k\right]
\end{align}



\section{Discussion}
\textbf{1. } Two propagators $\left[ \frac{\delta_{jk}r_i}{r^3} - \frac{3r_ir_jr_k}{r^5} \right]$ and $\left[ - \frac{3r_ir_jr_k}{r^5} \right]$ are used interchangeably in literature, but they are equivalent only for trace free tensors.

\textbf{2.} The stress generated by a point force is 
$$\sigma_{ij} = -p\delta_{ij} + \mu \left(u_{i,j}+u_{j,i}\right) = -\frac{3}{4\pi}\frac{r_ir_jr_k}{r^5}F_k. $$ 

\textbf{3. } Pay attention to the prefactors. The prefactors in STFMM3D and Blake 1971 for stress propagators to velocity and pressure seem to be both off by a factor of 2 and a negative sign.

\section{FMM}
In KIFMM the single layer could be used for M2M, M2L, L2L operations for both single layer and double layer potential. However the double layer potential Eq.~\ref{eq:doublelayerfmm} cannot be expressed by single layers if $D_{jk}$ is not trace free. In this case, we could set:
\begin{align}
	D_{jk} = S_{jk} + \frac{1}{3}D_{mm}\delta_{jk}
\end{align}  
where $S_{jk}$ is trace-free and $D_{mm}$ is the trace. In this way, Eq.~\ref{eq:doublelayerfmm} is converted to:
\begin{align}
		p& =\frac{1}{4\pi}\left(-3\frac{r_jr_k}{r^5} + \frac{\delta_{jk}}{r^3}\right)\left(S_{jk} + \frac{1}{3}D_{mm}\delta_{jk}\right), \quad u_i = \frac{1}{8\pi\mu}\left[ - \frac{3r_ir_jr_k}{r^5} \right] \left(S_{jk} + \frac{1}{3}D_{mm}\delta_{jk}\right)
\end{align}
It simplifies to:
\begin{align}
p& =\frac{1}{4\pi}\left(-3\frac{r_jr_k}{r^5}\right) S_{jk}, \quad u_i = \frac{1}{8\pi\mu}\left[ - \frac{3r_ir_jr_k}{r^5} \right]S_{jk} + \frac{1}{8\pi\mu}\left(-\frac{r_i}{r^3}\right)D_{mm}
\end{align}
The $S_{jk}$ part could be described by single layer sources, but the $D_{mm}$ term is not. Therefore we should extend the single layer kernel from 3 to 4 dimensions: $(F_x,F_y,F_z,D_{mm})$.
After the extension, the single layer kernel is:
\begin{align}\label{eq:singlelayerfmm}
	p=\frac{1}{4\pi} \frac{r_k}{r^3} F_k,\quad u_i = \frac{1}{8\pi\mu}\left(\frac{r^2\delta_{ij}+r_ir_j}{r^3} F_j -\frac{r_i}{r^3}D_{mm}\right)
\end{align}
Note that the trace of double layer $D_{mm}$ does not generate pressure. Therefore the pressure gradient is the same as those given in Section 4.2. Also the Laplacian of $u_i$ due to the extra $D_{mm}$ term is also zero. 

The velocity gradient due to $D_{mm}$ is:
\begin{align}
\dpone{u_i}{x_j} = \frac{1}{8\pi\mu}\left(\frac{3r_ir_j}{r^5}-\frac{\delta_{ij}}{r^3}\right)D_{mm}
\end{align}
The traction is then computed based on this.

\section{Summary: what are computed in the PVFMM Wrapper, exactly}
The Stokes single layer kernel is $4\times4$, from $(F_x,F_y,F_z,D_{mm})$ to $(p,u_x,u_y,u_z)$, as shown in Eq.~\ref{eq:singlelayerfmm}:
\begin{align}
p=\frac{1}{4\pi} \frac{r_k}{r^3} F_k,\quad u_i = \frac{1}{8\pi\mu}\left(\frac{r^2\delta_{ij}+r_ir_j}{r^3} F_j -\frac{r_i}{r^3}D_{mm}\right)
\end{align}

The double layer kernel is $9\times4$, from $(D_{xx},D_{xy},D_{xz},D_{yx},D_{yy},D_{yz},D_{zx},D_{zy},D_{zz})$ to $(p,u_x,u_y,u_z)$, as shown in Eq.~\ref{eq:doublelayerfmm}:
\begin{align} 
p& =\frac{1}{4\pi}\left(-3\frac{r_jr_k}{r^5} + \frac{\delta_{jk}}{r^3}\right)D_{jk}, \quad u_i = \frac{1}{8\pi\mu}\left[ - \frac{3r_ir_jr_k}{r^5} \right] D_{jk}
\end{align}
$D_{jk}$ here is arbitrary, not limited to trace-free cases, not limited to cases where $D_{jk}=v_kg_j$.

The gradient, Laplacian, and traction are computed with derivatives of these.


\end{document}
